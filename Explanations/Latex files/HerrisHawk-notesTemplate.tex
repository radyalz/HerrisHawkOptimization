\documentclass[12pt]{article}
\usepackage{geometry}
\usepackage{xepersian}
\settextfont{XB-nilo.TTF} % Persian font
\setlatintextfont{Vazirmatn.ttf} % Latin font
\newfontface\custompersianfont[Contextuals={WordInitial,WordFinal}]{XB-NILOOFARBDIT.TTF} % Another Persian font
% Adjust margins
\newfontface\customNumering{Times New Roman}
\geometry{a4paper, left=1in, right=1in}
\begin{document}

\section*{یادداشت‌هایی برای ارائه}
این صفحه را به عنوان یادداشت شخصی برای خودم نوشتم. گفتم شاید برای دیگران هم مفید باشد.

\subsection*{پیش‌ارائه}
بحث در مورد نکات جمع‌آوری‌شده و نمایش لینک‌های مرتبط.

\subsection*{مقدمه}
\begin{itemize}
    \item توضیح این موضوع که معمولاً ارائه‌ها را می‌توان به سه سؤال کلیدی خلاصه کرد: 
    ۱. چرا؟ 
    ۲. چطور؟ 
    ۳. چگونه؟
    
    \item اشاره به تحقیقات متعدد دکتر لوئیس لوفور درباره ضریب هوشی پرندگان، که در آن شاهین‌ها را به عنوان یکی از باهوش‌ترین پرندگان طبیعت معرفی کرده است.
    
    \item بررسی روش‌های شکار شاهین‌های هریس، با تأکید بر موضوعات زیر:
    \begin{itemize}
        \item مقایسه شاهین‌های هریس با پستاندارانی همچون شیر.
        \item ویژگی‌ها و تکنیک‌های شکار شاهین‌ها، از جمله توانایی آن‌ها در ردیابی، به دام انداختن، و حمله ناگهانی. بر اساس بررسی‌ها، این شاهین‌ها صبح زود به صورت گروهی حرکت می‌کنند و منطقه به منطقه به دنبال شکار می‌گردند. طعمه‌های معمول آن‌ها معمولاً خرگوش یا موش هستند.
        \item توضیح تکنیک‌های شکار نظیر \lr{Seven Kills} و ایجاد سردرگمی برای طعمه. همچنین اشاره به مزایای شکار گروهی، مانند خسته کردن طعمه.
    \end{itemize}
\end{itemize}
\subsection*{شرح موضوع مقاله}
\begin{itemize}
    \item ذکر سال انتشار مقاله.
    \item معرفی تمام نویسندگان مقاله.
    \item ارائه معرفی کوتاهی از خود مقاله.
\end{itemize}

\subsection*{شرح فرآیند}
\subsubsection*{مرحله اول: تعریف مسئله}
در ابتدای کار، مسئله به طور کامل تعریف شود.  
با توجه به شکل صفحه ۵، به سه موضوع زیر اشاره شود:
\begin{itemize}
    \item اکتشاف
    \item نحوه انتقال از اکتشاف به محاصره
    \item استثمار
\end{itemize}
پس از تعریف مقدماتی هر موضوع، به اسلایدهای بعدی مراجعه شود.

\subsubsection*{مرحله دوم: توضیح اکتشاف}
موضوعات زیر در این مرحله توضیح داده شود:
\begin{itemize}
    \item نحوه جست‌وجوی گروهی و استراتژی نشستن هر شاهین.
    \item نمایش معادله.
    \item توضیح عملکرد معادله اصلی.
\end{itemize}

\subsubsection*{مرحله سوم: توضیح فاز انتقال}
در فاز انتقال، به نحوه محاسبه انرژی طعمه اشاره شود و بحث شود که چگونه می‌توان از فاز اکتشاف به سمت استثمار منتقل شد.  
با توجه به اهمیت این صفحه، از نمایش معادله در صفحات بعد خودداری شود، زیرا ممکن است باعث ایجاد پرسش‌های بیشتری شود.

\subsubsection*{مرحله چهارم: توضیح استثمار}
صفحات ۸ تا ۱۰ برای توضیح این بخش اختصاص داده شود.  
در هر مورد زیر، ابتدا شرط اولیه اجرای استثمار را بیان کرده، سپس معادله را نمایش دهید و با توضیح فرآیند انجام‌شده توسط شاهین ادامه دهید.  
موارد و نکاتی که در این مرحله نیاز است به شرح زیر هستند:
\begin{itemize}
    \item محاصره نرم.
    \item محاصره سخت (نمایش و توضیح شکل را فراموش نکنید).
    \item محاصره نرم با شیرجه‌های سریع پیشرونده (بیشتر بر توضیح شکل تأکید کنید، زیرا معادله پیچیدگی درک بالایی داشت).
    \item محاصره سخت با شیرجه‌های سریع پیشرونده (مانند مورد قبل، بر توضیح شکل تأکید بیشتری داشته باشید، زیرا معادله فهم دشواری داشت).
\end{itemize}

\subsection*{نتیجه‌گیری و نمایش شبه‌کد}
با توجه به اینکه زمان حداکثر ده دقیقه است، اگر در صفحه ۱۱ زمان کافی وجود داشت، می‌توانید به معرفی و نمایش چند مقاله جمع‌آوری‌شده بپردازید.  
در غیر این صورت، صرفاً شبه‌کد را نشان داده و توضیح دهید.  
در نهایت، به پیچیدگی محاسباتی زمانی اشاره کنید و از توضیحات غیرضروری مانند تعریف \( N \) (تعداد شاهین‌ها)، \( T \) (حداکثر تعداد تکرار)، و \( D \) (بعد مسائل خاص) خودداری کنید.

\subsection*{پیشنهادات}
(در صورت نیاز، پیشنهادات خود را در این بخش ذکر کنید.)

\subsection*{تشکر و اتمام موضوع}
صرفاً تشکر کنید؛ این بخش نیازی به توضیحات اضافی ندارد.

\newpage 
\LTR

\section*{Notes for Presentation}
This page was written as personal notes for myself. I thought it might also be helpful for others.

\subsection*{Pre-Presentation}
Discuss the collected points and display relevant links.

\subsection*{Introduction}
\begin{itemize}
    \item Explain that most presentations can generally be summarized into three key questions: 
    \customNumering{1-} Why?  
    \customNumering{2-} How?  
    \customNumering{3-} What?  
    
    \item Mention the numerous studies by Dr. Louis Lefort on bird intelligence, which identify hawks as one of the most intelligent birds in nature.
    
    \item Review the hunting methods of Harris hawks, focusing on the following points:
    \begin{itemize}
        \item Comparison of Harris hawks with mammals like lions.
        \item Characteristics and hunting techniques of hawks, including their abilities to track, trap, and launch sudden attacks. Studies indicate that these hawks hunt in groups early in the morning, moving region to region in search of prey. Their typical prey includes rabbits or mice.
        \item Description of hunting strategies like \textit{Seven Kills} and disorienting prey. Also, discuss the benefits of group hunting, such as exhausting the prey.
    \end{itemize}
\end{itemize}

\subsection*{Paper Overview}
\begin{itemize}
    \item State the year of publication.
    \item Introduce all the authors of the paper.
    \item Provide a brief introduction to the paper itself.
\end{itemize}

\subsection*{Process Description}
\subsubsection*{Step \customNumering{1}: Defining the Problem}
Start by fully defining the problem.  
Based on the figure on page 5, refer to the following three topics:
\begin{itemize}
    \item Exploration
    \item Transition from exploration to encirclement
    \item Exploitation
\end{itemize}
After a preliminary definition of each topic, refer to the subsequent slides.

\subsubsection*{Step \customNumering{2}: Explaining Exploration}
The following topics should be explained in this phase:
\begin{itemize}
    \item Group searching methods and the landing strategy of each hawk.
    \item Display the equation.
    \item Explain the functionality of the main equation.
\end{itemize}

\subsubsection*{Step \customNumering{3}: Explaining the Transition Phase}
In the transition phase, discuss how prey energy is calculated and how the transition from exploration to exploitation occurs.  
Due to the importance of this slide, avoid displaying the equation on later slides, as it might raise additional questions.

\subsubsection*{Step \customNumering{4}: Explaining Exploitation}
Pages 8 to 10 are designated for explaining this section.  
For each of the following items, first state the initial condition for the exploitation, then display the equation and proceed with explaining what the hawk does during the process.  
The key points for this phase are as follows:
\begin{itemize}
    \item Soft encirclement.
    \item Hard encirclement (do not forget to display and explain the figure).
    \item Soft encirclement with fast progressive dives (focus more on explaining the figure, as the equation was difficult to understand).
    \item Hard encirclement with fast progressive dives (similar to the previous item, focus more on explaining the figure due to the equation's complexity).
\end{itemize}

\subsection*{Conclusion and Pseudocode Presentation}
Given the maximum allowed time of ten minutes, if there is sufficient time on page 11, you can introduce and display a few collected articles.  
Otherwise, simply present the pseudocode and provide explanations.  
Finally, discuss the computational time complexity and avoid unnecessary explanations such as defining \( N \) (number of hawks), \( T \) (maximum iterations), and \( D \) (specific problem dimension).

\subsection*{Suggestions}
(Include any suggestions here if needed.)

\subsection*{Acknowledgment and Closing Remarks}
Simply thank the audience; there is no additional topic to discuss in this section.


\end{document}