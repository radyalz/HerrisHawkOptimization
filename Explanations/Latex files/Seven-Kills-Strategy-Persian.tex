\documentclass{article}

\usepackage{geometry}
\usepackage{xepersian}
\settextfont{XB-nilo.TTF} % Persian font
\setlatintextfont{Vazirmatn.ttf} % Latin font
\newfontface\custompersianfont[Contextuals={WordInitial,WordFinal}]{XB-NILOOFARBDIT.TTF} % Another Persian font
% Adjust margins
\newfontface\customNumering{Times New Roman}
\geometry{a4paper, left=1in, right=1in}

\begin{document}

\setLTR
\section*{\customNumering{Harris Hawks Optimization: Explanation of Seven Kills}}

Harris Hawks Optimization (HHO) is a nature-inspired algorithm that mimics the **hunting behavior of Harris Hawks**. It uses several strategies to explore the solution space and find optimal solutions. One of the most distinctive features of HHO is its use of the **Seven Kills** strategy, which simulates the various **hunting tactics** employed by hawks. The Seven Kills represent different strategies in the optimization process, each aimed at improving the search and convergence process.

\subsection*{\customNumering{1. Soft Besiege Strategy}}
In the soft besiege strategy, the hawk adopts a **non-aggressive** approach to surround its prey, reflecting a gentle exploration of the solution space in the HHO algorithm. The algorithm searches for the optimal solution through gradual adjustments without drastic changes.

\subsection*{\customNumering{2. Hard Besiege Strategy}}
The hard besiege strategy involves **aggressive** tactics where the hawk forces its prey into a corner. Similarly, the algorithm aggressively explores the solution space to find the best possible solution quickly.

\subsection*{\customNumering{3. Multiple Pursuits}}
In this strategy, the hawk pursues **multiple targets** simultaneously. This mimics the algorithm’s ability to explore multiple candidate solutions at the same time, increasing its chances of finding the optimal solution.

\subsection*{\customNumering{4. Encirclement Strategy}}
The hawk uses **encirclement** tactics to trap its prey. In HHO, this reflects the algorithm’s ability to gradually surround the optimal solution by narrowing down the search space and refining the solutions.

\subsection*{\customNumering{5. Swooping Strategy}}
The swooping strategy is characterized by a **quick and sudden attack** on the prey. In optimization, this is similar to the algorithm’s ability to quickly converge on a solution once the search space is sufficiently narrowed down.

\subsection*{\customNumering{6. Surprise Strike Strategy}}
A hawk might use a **surprise strike** to catch its prey off guard. Similarly, the HHO algorithm sometimes uses unexpected moves or adjustments that lead to finding better solutions than expected.

\subsection*{\customNumering{7. Aggressive Strike Strategy}}
Finally, the aggressive strike strategy is the hawk’s **forceful attack** to secure its prey. In the HHO algorithm, this corresponds to the final push towards an optimal solution, using bold adjustments to refine the solution and avoid local optima.

\newpage 
\RTL

\section*{\custompersianfont{بهینه‌سازی شاهین هریس: توضیح مفهوم هفت قتل}}
بهینه‌سازی هریس هاوک (HHO) یک الگوریتم الهام‌گرفته از طبیعت است که رفتار شکار **هاوک‌های هریس** را شبیه‌سازی می‌کند. این الگوریتم از چندین استراتژی برای کاوش در فضای راه‌حل و یافتن راه‌حل‌های بهینه استفاده می‌کند. یکی از ویژگی‌های برجسته HHO استفاده از استراتژی **هفت قتل** است که تاکتیک‌های مختلف **شکار** هاوک‌ها را شبیه‌سازی می‌کند. هفت قتل نمایانگر استراتژی‌های مختلف در فرایند بهینه‌سازی هستند که هدف آن‌ها بهبود فرایند جستجو و همگرایی است.

\subsection*{۱. استراتژی محاصره نرم}
در استراتژی محاصره نرم، هاوک رویکردی **غیرتهاجمی** را برای محاصره طعمه خود اتخاذ می‌کند که بازتابی از کاوش تدریجی فضای راه‌حل در الگوریتم HHO است. این الگوریتم بدون تغییرات شدید، با تنظیمات تدریجی به دنبال بهترین راه‌حل می‌گردد.

\subsection*{۲. استراتژی محاصره سخت}
استراتژی محاصره سخت شامل تاکتیک‌های **تهاجمی** است که در آن هاوک طعمه خود را به گوشه‌ای هدایت می‌کند. مشابه آن، الگوریتم به طور تهاجمی فضای جستجو را برای یافتن سریع‌ترین راه‌حل ممکن کاوش می‌کند.

\subsection*{۳. پیگیری‌های متعدد}
در این استراتژی، هاوک به طور همزمان **چندین هدف** را تعقیب می‌کند. این شبیه به توانایی الگوریتم برای کاوش در چندین راه‌حل کاندیدا به طور همزمان است که شانس یافتن بهترین راه‌حل را افزایش می‌دهد.

\subsection*{۴. استراتژی محاصره}
هاوک از تاکتیک‌های **محاصره** برای گیر انداختن طعمه استفاده می‌کند. در HHO، این به توانایی الگوریتم برای محاصره تدریجی راه‌حل بهینه اشاره دارد، که فضای جستجو را محدود کرده و راه‌حل‌ها را تصحیح می‌کند.

\subsection*{۵. استراتژی پرواز سریع}
استراتژی پرواز سریع با **حمله ناگهانی و سریع** به طعمه مشخص می‌شود. در بهینه‌سازی، این مشابه توانایی الگوریتم برای همگرایی سریع به سمت یک راه‌حل پس از محدود شدن فضای جستجو است.

\subsection*{۶. استراتژی حمله غافلگیرانه}
هاوک ممکن است از **حمله غافلگیرانه** برای غافلگیر کردن طعمه خود استفاده کند. مشابه آن، الگوریتم HHO گاهی از حرکات یا تنظیمات غیرمنتظره استفاده می‌کند که به یافتن راه‌حل‌های بهتر از آنچه پیش‌بینی می‌شود، منجر می‌شود.

\subsection*{۷. استراتژی حمله تهاجمی}
در نهایت، استراتژی حمله تهاجمی **حمله‌ای قوی** است که طعمه را به طور قطع به دام می‌اندازد. در الگوریتم HHO، این معادل حرکت نهایی به سمت یک راه‌حل بهینه است که از تنظیمات جسورانه برای تصحیح راه‌حل و جلوگیری از گیر کردن در نقاط بهینه محلی استفاده می‌کند.

\end{document}
